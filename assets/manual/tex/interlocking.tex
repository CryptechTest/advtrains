\section{Interlocking and line automation}\label{s:interlocking}

Interlocking is a set of equipments employed to prevent train collisions while allowing trains to go to their destinations. This chapter will explain some basic concepts and includes a task at the end that you can try to do yourself.

Please note that this chapter is also intended to be used as a reference, so some things may be explained in a way that is not easy to understand at first. If that is the case, the following links may be helpful, albeit possibly outdated:
\begin{itemize}
\item The online interlocking guide: \url{https://advtrains.de/interlocking}
\item Blockhead's Youtube video showing a simple three-station setup (see section \ref{s:ilbasicsetup}): \url{https://www.youtube.com/watch?v=ylG1vHj4zjg}
\end{itemize}

It is also recommended for new players to read through the entire section (or at least the first few subsections) to understand various concepts of the interlocking system before following the instructions in the subsections.

\subsection{TCBs}\label{s:iltcbs}
Track circuit breaks (TCBs) are nodes that can be assigned to two-way tracks and indicate the limits of track sections. TCBs have two sides - each corresponding to a direction of the track that the TCB is assigned to - that can be assigned to two track sections.

To assign a TCB to a track:
\begin{itemize}
\item Place a TCB, ideally adjacent to the track it will be assigned to.
\item Right-click the TCB.
\item You will be told in the chat to punch a track to assign the TCB to. Punch the track to assign the TCB to - this track will, in later subsections, be the boundary of track sections.
\item A TCB marker showing the two sides (A and B) will appear on the track you have assigned the TCB to. This will be explained in further detail below.
\end{itemize}

After assigning the TCB, you can right click it to open the TCB formspec. This will be used in the following sections. Please note that the TCB formspec has a section for side A and one for side B - when following the instructions in the following subsections, make sure you click the button for the right TCB side.
The two sides of the TCB are assigned based on the orientation of the tracks. In particular, straight tracks with different orientations can still appear visually the same. For this reason, most graphs in the rest of this chapter will use cardinal directions instead of actual TCB sides, or omit TCB sides if these are irrelevant.

For new players, it is recommended to set up all TCBs before creating new track sections.

\subsection{Track sections}\label{s:ilts}

In terms of interlocking, track sections are segments of tracks that can be occupied by a train or reserved for a train to pass. The following graph shows TCBs and sections along a line segment. For simplicity, letters are used instead of actual coordinates and IDs.

\begin{centeredtikzpicture}
  \draw[<->] (-2,0) node [left] {W} -- (7,0) node [right] {E};
  % first TCB
  \draw (0,-0.5) -- ++(0,2) node [above] {TCB assigned to $P$};
  \draw (0,1) node [left] {Section $a$} node [right] {Section $b$};
  \draw[->,Red] (0,0.5) -- +(-1.5,0) node [left] {A};
  \draw[->,ForestGreen] (0,0.5) -- +(1.5,0) node [right] {B};
  %second TCB
  \draw (5,-0.5) -- ++(0,2) node [above] {TCB assigned to $Q$};
  \draw (5,1) node [left] {Section $b$} node [right] {EOI};
  \draw[->,Red] (5,0.5) -- +(-1.5,0) node [left] {A};
  \draw[->,ForestGreen] (5,0.5) -- +(1.5,0) node [right] {B};
\end{centeredtikzpicture}

Section $a$ begins at the side A of $P$ (sometimes written as $P/\mathrm{A}$) and extends to the west beyond the graph. Section $b$ is limited by $P/\mathrm{B}$ and $Q/\mathrm{A}$, effectively spanning the line segment $\overline{PQ}$. $Q/\mathrm{B}$ is not part of any track section and is therefore known as \textit{end of interlocking} or \textit{EOI}, as the track beyond it is not part of the interlocking system.

The graph below shows a track segment where $P$, $Q$, and $R$ form a section around the turnout $T$. Notice that a train entering from $Q$ cannot directly reach $R$ without reversing (or vice versa),  while a train entering from $P$ can reach both $Q$ and $R$.

\begin{centeredtikzpicture}
  \draw[<->] (0,0) -- (7,0);
  \draw[->] (3,0) .. controls(4,0) .. (5,1) -- (7,3);
  \foreach \x/\y/\c in {1/0/$P$,6/0/$Q$,6/2/$R$,3/0/$T$} {
    \node[circle,fill=white,draw=black] at (\x,\y) {\c};
  }
  \node at (-0.5,0) {W};
  \node at (7.5,0) {E};
\end{centeredtikzpicture}

To create a track section, click on ``Create new track section'' in the TCB formspec. This should create a new track section and add the TCB side to the track section. Other track sides that face the track side you created the TCB with will also be added to the track section. In the graph above, creating a track section at the west side of $Q$ will automatically add the south-west side of $R$ and the east side of $P$ to the track section.

After the track section is created, you can click on ``Show track section'' to open the track section formspec, or ``Remove from section'' to move the TCB side from the track section.

In some cases, such as crossings, you may need to manually add TCB sides to a track section. This can only be done by creating a track section for the TCB side and then merging the two track sections.

To merge two track sections:
\begin{itemize}
\item Click ``Join into other section'' in the formspec of the section that needs to be merged.
\item Click ``Join with \var{section}'' in the formspec of the section to merge the other section into, where \var{section} is the ID of the section that needs to be merged.
\end{itemize}
To abort the procedure above, click on the ``X'' button on the right.

The track section formspec also has a button labeled ``Dissolve section''. Clicking on that button will remove the track section entirely.

In most cases, the two sides of the TCB should belong to different sections, or at least one side belongs to EOI. If this is not the case (i.e., both sides of the TCB belong to the same track section), it likely means that the presence of the TCB is irrelevant for the track section. With the current implementation of the section occupation system, this is problematic as the section behind the train is always freed, even when the train passes a TCB with both sides belonging to the same section. If you run into such a situation, dissolve the track section and make sure that every TCB is assigned to a track and delimits the track sections that the two sides of the TCB are assigned to, and then create the sections again.

\subsection{Signals}\label{s:ilsignals}

Each TCB side can have a signal assigned to the side. The signal will then indicate whether the train is allowed to enter the section to which the same TCB side is assigned to. Signals need to be set up in order to be able to set up routes. In the previous graph, in order to let trains move from $P$ to $Q$ or $R$, a signal needs to be set up on the side of $P$ that faces the track section (i.e. the east side of $P$). For an overview of the signals available in \advtrains, refer to section \ref{s:signals}. Signals should conventionally face the opposite direction of the side of the TCB so that the driver can see the signals.

Every signal can optionally have an influence point. This is the point where the aspect of the signal becomes effective, and should be located before the train passes the TCB side that the signal is assigned to. The influence point for signals that are independent of track sections is irrelevant, but should conventionally be close to the signal.

To assign a signal to a TCB:
\begin{itemize}
\item Click on ``Assign a signal'' in the TCB formspec.
\item You will be asked to punch a signal. Punch the signal to assign to the TCB side.
\item If you are prompted to set the influence point, you can do so by left clicking the track you want to assign the influence point to while looking in the same direction that the train drives in.
\end{itemize}

You can also set the influence point of a signal without assigning it to the TCB. To do so, right click the signal while holding the Aux1 key. For signal signs, right-clicking the signal will bring up the same prompt.

After assigning a signal to a TCB side, you can right click it to open up the signal formspec. The ``Influence point'' button will open a formspec let allows you to change or remove the influence point. The ``Unassign signal'' button will unassign the signal from the TCB side while keeping the influence point (if present).

\subsection{Routes}\label{s:ilroutes}

Routes contain information on where a train goes to, which section should be reserved for the train to pass, and how certain components, such as turnouts, should be set up. Routes are bound to TCB sides, but they usually need to start at ones with a signal assigned. In most cases, routes should also end in such a way that a train leaving the route can immediately enter the next route, with the most common exception being situations where dead ends are involved, such as train yards and termini where the line physically ends (e.g. Stuttgart Hauptbahnhof).

A relatively simple example can be seen in the graph below.

\begin{centeredtikzpicture}
  \draw[<->] (-0.5,0) -- (8.5,0);
  \draw[->] (4,0) .. controls +(1,0) .. ++(2,1) -- ++(2.5,2.5);
  \foreach \x/\y/\c in {0/0/$P_{1}$,2/0/$P_{2}$, 4/0/$T$,6/0/$Q_{2}$,8/0/$Q_{1}$,6/1/$R_{2}$,8/3/$R_{1}$} {
    \draw [fill=white] (\x,\y) circle [radius=0.3];
    \node at (\x,\y) {\c};
  }
  \foreach \x/\y in {0/0,2/0,8/0,8/3} {
    \pic [xscale=-1,shift={(0,0.3)}] at (\x,\y) {signal=red};
  }
  \foreach \x/\y in {6/0,6/1} {
    \pic [shift={(0,0.3)}] at (\x,\y) {tcb};
  }
  \node at (-1,0) {W};
  \node at (9,0) {E};
\end{centeredtikzpicture}

A train going from west to east (or north-east) would go through the following points in order:
\begin{itemize}
\item $P_{1} \to P_{2} \to T \to Q_{2} \to Q_{1}$
\item $P_{1} \to P_{2} \to T \to R_{2} \to R_{1}$
\end{itemize}

Note that the train goes through $\overline{P_{1}P_{2}}$ in both cases. $\overline{P_{2}T}$ is also shared, but $T$ needs to be set up based on the destination and the point cannot be assigned to a TCB. It is also encouraged to set up routes in a way that trains do not occupy the section containing any turnout while the train waits for a red signal (or stops in any way). This means that three routes can be set up for this example:
\begin{itemize}
\item $P_{1} \to P_{2}$
\item \(P_{2} \to \text{(locking $T$ to point at $Q_{2}$)} \to Q_{2} \to Q_{1}\)
\item \(P_{2} \to \text{(locking $T$ to point at $R_{2}$)} \to R_{2} \to R_{1}\)
\end{itemize}

To create a route:
\begin{enumerate}
\item Click on ``New route'' in the formspec of the signal assigned to the starting TCB side.
\item Set up passive components (e.g. turnouts) in the section appropriately and lock them by punching, if necessary.
\item Punch the next TCB that the train should pass. The track that the TCB is assigned to should have a marker showing ``END ROUTE''.
\item If the following track section is the last track section in the route and does not require any setup require in step 2 or have more than two TCB sides, click ``Finish route at the end of NEXT section''.
\item If the route leads to the end of the physical line, click ``Finish route at the end of NEXT section''.
\item If the route ends at the TCB, click ``Finish route HERE''.
\item If the route continues beyond the TCB, click ``Advance to next route section'' and repeat the above steps, starting from step 2.
\end{enumerate}

You can also click ``Finish route at the end of NEXT section'' at the starting TCB. This can be useful for the $\overline{P_1P_2}$ route in the previous graph.

If the TCB is not considered suitable for route continuation, please check that the side of the TCB from which the train passes the TCB is part of a track section and that the train can reach there without passing a point assigned to another TCB. If you see ``Advancing over next section is impossible at this place. End of interlocking.'', please check that the side of the TCB corresponding to the driving direction is part of a track section.

\begin{centeredtikzpicture}
  \draw [->] (-0.5,1) node [left] {Finish route HERE} -- (4.5,1);
  \draw [->,blue,very thick] (0,1) -- (2,1);
  \draw [->] (-0.5,0) node [left] {Finish route at the end of NEXT section}-- (4.5,0);
  \draw [->,blue,very thick] (0,0) -- (4,0);
  \foreach \y in {1,0} {
    \draw (0,\y) pic [xscale=-1] {signal=red} node [below] {$P$};
    \foreach \x/\l in {2/Q,4/R} {
      \draw (\x,\y) pic {tcb} node [below] {$\l$};
    }
  }
\end{centeredtikzpicture}

After finishing the route, you are prompted for the name of the route. It is recommended to use a sensible name. You should then see the route formspec. If you don't, click on the name of the route you just created and click ``Edit route'', and proceed to the next section regarding ARS.

\subsection{Automatic routesetting}\label{s:ilars}
Automatic routesetting (ARS) is a method of choosing a route based on a set of matching patterns. In the previous section, you learned to set up a route, and there is a text input box where you can enter the ARS rules. This area is empty by default, which means that the route is not selected in any case.

ARS rules are delimited by a newline. Each line can be one of the following:
\begin{apidoc}{ARS rules}
\item \apipat{\# \var{comment}} Comment
\item \apipat{LN \var{line}} Matches trains with the exact line name \var{line}.
\item \apipat{RC \var{routing code}} Matches trains that contain the routing code \var{routing code}.
\item \apipat{!LN \var{line}} Matches trains with a different line name than \var{line}.
\item \apipat{!RC \var{routing code}} Matches trains that do not contain the routing code \var{routing code}.
\item \apipat{*} Matches all trains
\end{apidoc}

A whitespace is required before the argument.

ARS rules are designed to short-circuit. The route is selected if a train matches any of the rules.

In some cases, you may want to disable ARS. To do so, click on ``Disable ARS'' in the signal formspec. Clicking on the same button will enable ARS.

\subsection{Automatic working}\label{s:ilaw}
Automatic working is a system that sets the route after the train passes the signal. To enable automatic working, click on ``Enable Automatic Working'' in the signal formspec after the route is set. Clicking on the same button again will disable automatic working.

\subsection{Station/stop tracks}\label{s:stoprail}
Station/stop tracks (sometimes simply called \textit{station tracks}) are special tracks that register a station where a train should stop. The interface of the formspec should be self-explanatory, with a few things to notice:
\begin{itemize}
\item You need to click ``Save'' after changing the settings.
\item The station code is used to identify the station. The station name is shared among all tracks with the same station code (not vice versa).
\item Station tracks disable ARS for the specific train when the train arrives at the station and enables ARS on the train before departure.
\end{itemize}

\subsection{Considerations for interlocking}\label{s:ilconsiderations}

The previous sections were mainly theoretical in that the sections mostly introduced new concepts or described how to do things. This section will focus on the practical part of interlocking, in particular certain things to consider when setting up interlocking on a rail line.

\subsubsection{Junctions}\label{s:iljunctions}

One of the most important considerations when setting up interlocking at a junction is to make sure that multiple trains can go through the junction at the same time when possible.

As an exercise in section \ref{s:xings}, you were asked to build a T junction. An example is shown below if you are stuck.\footnote{Please note that, as in the rest of the manual, the curves in the diagrams only approximate their counterparts in the game. It is up to the reader to figure out which track to use.} Try to interlock the junction based on what you have learned in the last few sections.

\begin{centeredtikzpicture}[ultra thick,scale=0.7]
  \draw [gray,thin,shift={(-0.5,-0.5)}] (0,0) grid (16,11);
  \draw [<->] (0,10) -- ++(15,0);
  \draw [<->] (0,7) -- ++(15,0);
  \draw [->] (1,10) .. controls +(1,0) .. ++(2,-0.5) .. controls +(1,-0.5) .. ++(2,-1.5) -- ++(2,-2) .. controls +(1,-1) .. ++(1.5,-2) .. controls +(0.5,-1) .. ++(0.5,-2) -- ++(0,-2);
  \draw (1,7) .. controls +(1,0) .. ++(2,-0.5) .. controls +(1,-0.5) .. ++(1.5,-1) .. controls +(0.5,-0.5) .. ++(1,-1.5) .. controls +(0.5,-1) .. ++(0.5,-2);
  \draw [->] (14,10) .. controls +(-1,0) .. ++(-2,-0.5) .. controls +(-1,-0.5) .. ++(-2,-1.5) -- ++(-2,-2) .. controls +(-1,-1) .. ++(-1.5,-2) .. controls +(-0.5,-1) .. ++(-0.5,-2) -- ++(0,-2);
  \draw (14,7) .. controls +(-1,0) .. ++(-2,-0.5) .. controls +(-1,-0.5) .. ++(-1.5,-1) .. controls +(-0.5,-0.5) .. ++(-1,-1.5) .. controls +(-0.5,-1) .. ++(-0.5,-2);
\end{centeredtikzpicture}

The lazy method would be to set up the entire junction as a single track section - you only need 6 TCBs for that. However, a train passing through would occupy the entire junction, even when two trains could use the junction at the same time - for example, a train going from east to west and another one from west to east. This can be solved by a slightly more complicated setup with four sections, each shown below with a different color:

\begin{centeredtikzpicture}[ultra thick]
  \draw [<-,red] (-3,2) -- (-1.5,2) arc [start angle=90,end angle=60,radius=2];
  \draw [<-,red] (3,2) -- (1.5,2) arc [start angle=90,end angle=120,radius=2];
  \draw [red] (-1.5,2) -- (1.5,2);
  \draw [blue] (60:2) ++(-1.5,0) arc [start angle=60,end angle=15,radius=2];
  \draw [blue] (120:2) ++(1.5,0) arc [start angle=120,end angle=165,radius=2];
  \draw [blue] (-0.75,1) -- (0.75,1);
  \draw [<->,orange] (-3,1) -- ++(1.5,0) arc [start angle=90,end angle=0,radius=1] -- +(0,-1.5);
  \draw [orange] (-1.5,1) -- ++(0.75,0);
  \draw [orange] (-0.5,0) arc [start angle=180,end angle=165,radius=2];
  \draw [<->,green] (3,1) -- ++(-1.5,0) arc [start angle=90,end angle=180,radius=1] -- +(0,-1.5);
  \draw [green] (1.5,1) -- ++(-0.75,0);
  \draw [green] (0.5,0) arc [start angle=0,end angle=15,radius=2];
\end{centeredtikzpicture}

It is also possible to build the junction on multiple levels. This is known as grade separation. In real life, grade-separated railway junctions are not as common as junctions for road vehicles, mostly because of the larger curve radii and slopes that are notably less steep (for example, the steepest slope of the Schwarzwaldbahn in Baden, which has an altitude between 384m and 832m, is only 2\%). As an exercise, you can try to make the T junction grade-separated.

\subsubsection{Track capacity and deadlock}\label{s:ilcapacity}
You can not infinitely add trains to a line. If you do, you will end up with a deadlock, where every train is stuck at a red light, waiting for the previous train to clear the track section ahead, while the previous train is also stuck at a red light, waiting for the train ahead of it to leave the section ahead, and so on:

\begin{centeredtikzpicture}
  \draw (0,1) arc [start angle=90,end angle=270,radius=0.5] -- (8.5,0) arc [start angle=270,end angle=450,radius=0.5] -- cycle;
  \foreach \i in {0,2,...,6} {
    \draw[->,very thick,blue] (\i,0) ++(0.5,0) -- +(1,0);
    \pic at (\i,0) [shift={(2,0)},xscale=-1] {signal=red};
    \draw[<-,very thick,blue] (\i,1) ++(1,0) -- +(1,0);
    \pic at (\i,1) [shift={(0.5,0)}] {signal=red};
  }
\end{centeredtikzpicture}

This problem can be trivially solved by removing a train and, if necessary, resetting the track section that the train previously occupied (this can be done in the track section formspec):

\begin{centeredtikzpicture}
  \draw (0,1) arc [start angle=90,end angle=270,radius=0.5] -- (8.5,0) arc [start angle=270,end angle=450,radius=0.5] -- cycle;
  \foreach \i in {0,2,...,6} {
    \draw[->,very thick,blue] (\i,0) ++(0.5,0) -- +(1,0);
    \pic at (\i,1) [shift={(0.5,0)}] {signal=red};
  }
  \foreach \i in {2,4,...,6} {
    \pic at (\i,0) [xscale=-1] {signal=red};
    \draw[->,very thick,blue] (\i,1) -- +(-1,0);
  }
  \pic at (8,0) [xscale=-1] {signal=green};
\end{centeredtikzpicture}

However, as you may notice, only the train at the bottom-right can move; other trains have to wait for the trains ahead to leave the section. The ideal situation is that the section length and the number of trains are decided in such a way that trains do not have to stop or slow down between stations.

When actually building a rail line, signals are spaced away much further than in the illustration above. On servers, the distance between signals typically range between about 50 nodes to a few hundred, depending on the speed at which trains run on the line and the number of trains using the line. The distance between signals in station areas usually depend on the size of the station.

Single-track sections can also be a source of deadlocks:

\begin{centeredtikzpicture}[scale=0.6]
  \draw [<-] (-2.5,1) -- (-2,1) .. controls (-1,1) .. (-0.5,0.5) .. controls (0,0) .. (1,0) -- (3,0) .. controls (4,0) .. (4.5,0.5) .. controls (5,1) .. (6,1) -- (8,1) .. controls (9,1) .. (9.5,0.5) .. controls (10,0) .. (11,0) -- (13,0);
  \draw [<-] (-2.5,0) -- (13,0);
  \draw (13,-0.1) -- (13,0.1);
  \foreach \x/\y/\s in {-2/0/-1,1/0/1,3/0/-1,6/1/1,8/0/-1,11/0/1} {
    \pic at (\x,\y) [xscale=\s] {signal=red};
  }
  \foreach \x/\y in {-2/1,6/0,8/1} {
    \pic at (\x,\y) {tcb};
  }
  \foreach \x/\y/\d in {1.5/0/->,6.5/1/<-,6.5/0/->,11.5/0/<-} {
    \draw [\d,very thick,blue] (\x,\y) -- +(1,0);
  }
\end{centeredtikzpicture}

With single-track sections, routes should generally be set up in a way that the train does not stop in single track sections. Exceptions include the end of track, where the train has to stop and reverse:

\begin{centeredtikzpicture}[scale=0.6]
  \draw [<-] (-2.5,1) -- (-2,1) .. controls (-1,1) .. (-0.5,0.5) .. controls (0,0) .. (1,0) -- (3,0) .. controls (4,0) .. (4.5,0.5) .. controls (5,1) .. (6,1) -- (8,1) .. controls (9,1) .. (9.5,0.5) .. controls (10,0) .. (11,0) -- (13,0);
  \draw [<-] (-2.5,0) -- (13,0);
  \draw (13,-0.1) -- (13,0.1);
  \foreach \x/\y/\s in {-2/0/-1,8/0/-1,11/0/1} {
    \pic at (\x,\y) [xscale=\s] {signal=red};
  }
  \pic at (6,1) {signal=green};
  \foreach \x/\y in {-2/1,1/0,3/0,6/0,8/1} {
    \pic at (\x,\y) {tcb};
  }
  \foreach \x/\y/\d in {6.5/1/<-,6.5/0/->,11.5/0/<-} {
    \draw [\d,very thick,blue] (\x,\y) -- +(1,0);
  }
\end{centeredtikzpicture}

Notice that the deadlock is prevented at the cost of lower capacity of the line.

\subsubsection{Short routes}
In some cases, there may be setups with very short routes:

\begin{centeredtikzpicture}
  \draw [->] (0,0) -- (4.5,0);
  \draw [very thick, blue, ->] (0.5,0) -- (1.5,0);
  \pic at (2,0) [xscale=-1] {signal=green};
  \node [below] at (2,0) {$P$};
  \pic at (4,0) [xscale=-1] {signal=green};
  \node [below] at (4,0) {$Q$};
\end{centeredtikzpicture}

In the setup illustrated above, if $\overline{PQ}$ is short enough, or, more specifically, if $\overline{PQ}$ is shorter than the braking distance of the train at $P$, the signal at $Q$ will be triggered to set a route.

If short routes are desired, such as at stations, it is generally recommended to disable ARS for the train, such as by using the \texttt{A0} ATC command, and keep some distance between the influence point of the signal and the point at which the train is expected to stop. Alternatively, you can also limit the speed of the train to trigger the routesetting system later, but this is not the ideal solution in most use cases of short routes.

\subsection{Basic three-station setup}\label{s:ilbasicsetup}
Now that you have learned to use interlocking, create a simple three-station setup shown below.
\begin{centeredtikzpicture}[scale=0.6]
  \draw (0,0) -- (3,0) .. controls (4,0) .. (4.5,0.5) .. controls (5,1) .. (6,1) -- (8,1) .. controls (9,1) .. (9.5,0.5) .. controls (10,0) .. (11,0) -- (14,0);
  \draw (0,0) -- (14,0);
  \node at (1.5,0.5) {Station A};
  \node at (7,0.5) {Station B};
  \node at (12.5,0.5) {Station C};
\end{centeredtikzpicture}

If you are stuck at some point or want something to compare your set up to, you can watch Blockhead's introduction video, where he creates a three-station setup. The link is provided at the beginning of this chapter.

%%% Local Variables:
%%% TeX-master: "a4manual"
%%% End: